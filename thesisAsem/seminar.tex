%%%%%%%%%%%%%%%%%%%%%%%%%%%%%%%%%%%%%%%%%
% Beamer Presentation
% LaTeX Template
% Version 1.0 (10/11/12)
%
% This template has been downloaded from:
% http://www.LaTeXTemplates.com
%
% License:
% CC BY-NC-SA 3.0 (http://creativecommons.org/licenses/by-nc-sa/3.0/)
%
%%%%%%%%%%%%%%%%%%%%%%%%%%%%%%%%%%%%%%%%%

%----------------------------------------------------------------------------------------
%	PACKAGES AND THEMES
%----------------------------------------------------------------------------------------

\documentclass{beamer}

\mode<presentation> {

% The Beamer class comes with a number of default slide themes
% which change the colors and layouts of slides. Below this is a list
% of all the themes, uncomment each in turn to see what they look like.

%\usetheme{default}
%\usetheme{AnnArbor}
%\usetheme{Antibes}
%\usetheme{Bergen}
\usetheme{Berkeley}
%\usetheme{Berlin}
%\usetheme{Boadilla}
%\usetheme{CambridgeUS}
%\usetheme{Copenhagen}
%\usetheme{Darmstadt}
%\usetheme{Dresden}
%\usetheme{Frankfurt}
%\usetheme{Goettingen}
%\usetheme{Hannover}
%\usetheme{Ilmenau}
%\usetheme{JuanLesPins}
%\usetheme{Luebeck}
%\usetheme{Madrid}
%\usetheme{Malmoe}
%\usetheme{Marburg}
%\usetheme{Montpellier}
%\usetheme{PaloAlto}
%\usetheme{Pittsburgh}
%\usetheme{Rochester}
%\usetheme{Singapore}
%\usetheme{Szeged}
%\usetheme{Warsaw}

% As well as themes, the Beamer class has a number of color themes
% for any slide theme. Uncomment each of these in turn to see how it
% changes the colors of your current slide theme.

%\usecolortheme{albatross}
%\usecolortheme{beaver}
%\usecolortheme{beetle}
%\usecolortheme{crane}
%\usecolortheme{dolphin}
%\usecolortheme{dove}
%\usecolortheme{fly}
%\usecolortheme{lily}
%\usecolortheme{orchid}
%\usecolortheme{rose}
%\usecolortheme{seagull}
%\usecolortheme{seahorse}
%\usecolortheme{whale}
%\usecolortheme{wolverine}

%\setbeamertemplate{footline} % To remove the footer line in all slides uncomment this line
%\setbeamertemplate{footline}[page number] % To replace the footer line in all slides with a simple slide count uncomment this line

%\setbeamertemplate{navigation symbols}{} % To remove the navigation symbols from the bottom of all slides uncomment this line
}

\usepackage{graphicx} % Allows including images
\usepackage{booktabs} % Allows the use of \toprule, \midrule and \bottomrule in tables

%----------------------------------------------------------------------------------------
%	TITLE PAGE
%----------------------------------------------------------------------------------------

\title[Heterogeneous Scheduling]{Work Scheduling on Heterogeneous Systems}

\author{Gabriele Keller (supervisor) \and Edward Pierzchalski}
\institute[UNSW] % Your institution as it will appear on the bottom of every slide, may be shorthand to save space
{
University of New South Wales \\ % Your institution for the title page
\medskip
\textit{e.pierzchalski@unsw.edu.au} % Your email address
}
\date{\today} % Date, can be changed to a custom date

\begin{document}

\begin{frame}
\titlepage % Print the title page as the first slide
\end{frame}

\begin{frame}
\frametitle{Overview} % Table of contents slide, comment this block out to remove it
\tableofcontents % Throughout your presentation, if you choose to use \section{} and \subsection{} commands, these will automatically be printed on this slide as an overview of your presentation
\end{frame}

%----------------------------------------------------------------------------------------
%	PRESENTATION SLIDES
%----------------------------------------------------------------------------------------

\section{Heterogeneous Programming}

\begin{frame}
\frametitle{Why Heterogeneous Programming}
\begin{itemize}
\item People do math, simulations, and stream processing
  \begin{itemize}
   \item We can parallelise these!
  \end{itemize}

\item CPUs are becoming tiny graphics cards
\item GPUs are becoming tiny CPU swarms
\end{itemize}
\end{frame}

\begin{frame}
\frametitle{Why Not Heterogeneous Programming}
\begin{itemize}
\item Distributing work is fiddly (thesis worthy!)
\item Heterogeneous code is difficult, ugly, and imperative
\item Hard to compose solutions
  \begin{itemize}
    \item If I can write an optimised GPU kernel for maps, and another for prefix sums, can I `fuse' them together while maintaining performance?
  \end{itemize}
\end{itemize}
\end{frame}

\section{Accelerate}

\begin{frame}
\frametitle{Accelerate: An Array DSL}
\begin{itemize}
  \item A deeply embedded domain-specific language (the domain is array computation)
  \item Embedded in Haskell
    \begin{itemize}
      \item Lots of fancy type shenanigans to enforce semantics
    \end{itemize}
  \item Solves the `difficult, ugly, and imperative' problem
  \item Doesn't solve the `distributing work' problem
    \begin{itemize}
      \item Until recently!
    \end{itemize}
\end{itemize}
\end{frame}

\begin{frame}
\frametitle{Data Fission (by Newton, Holk, and McDonell)}
\begin{itemize}
  \item Added two nodes to Accelerate: \texttt{split} and \texttt{concat}
  \item Arrays are fissioned after operator fusion optimisations
  \item Initial algorithm: fission arrays into constant number of fragments, allocate each fragment to a device
\end{itemize}
\end{frame}

\section{Scheduling}

\begin{frame}
\Huge{\centerline{Can we do better?}}
\end{frame}

\begin{frame}
\frametitle{Kinds of Scheduling}
\begin{itemize}
  \item Static Scheduling: 
    \begin{itemize}
      \item Arrays are fissioned into constant number of work fragments
      \item Fragments are scheduled on devices according to data dependency
    \end{itemize}
  \item Dynamic Scheduling:
    \begin{itemize}
      \item Use runtime information to choose fragmentation and device scheduling
    \end{itemize}
\end{itemize}
\end{frame}

\begin{frame}
\frametitle{What Has Been Tried}
\begin{itemize}
  \item Machine learning on source code (Grewe and O'Boyle)
  \item Binary analysis of array indexing (Lee et. al.)
  \item Statically split and allocate (Newton, Holk, and McDonell)
  \item Runtime performance analysis (Wang et. al.)
\end{itemize}
\end{frame}

\begin{frame}
\frametitle{Speedy Dynamic Scheduling}
\begin{itemize}
  \item Devices perform differently under different workloads
  \item Use run-time profiling
  \item Do too little and you can't exploit device/workload differences
  \item Do too much and you get swamped by synchronisation overheads (both time and memory)
  \item Try something in-between!
\end{itemize}
\end{frame}

\section{Work Plan}

\begin{frame}
\frametitle{Main Goals}
\begin{itemize}
  \item Implement dynamic fissioning, dynamic allocation
    \begin{itemize}
      \item New \texttt{accelerate-backend-kit} already set up groundwork
      \item Compare scheduling algorithms
      \item Investigate interaction of fission and fusion
    \end{itemize}
  \item Small-array optimisations
  \item Benchmarking
\end{itemize}
\end{frame}

\begin{frame}
\frametitle{Extensions}
\begin{itemize}
  \item Improve array concatenation
  \item LLVM support for operations
  \item Device affinity for dynamic allocation algorithm
\end{itemize}
\end{frame}

\section{Questions}

\begin{frame}
\Huge{\centerline{Question Time!}}
\end{frame}

%----------------------------------------------------------------------------------------

\end{document} 